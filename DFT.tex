\chapter{Density Functional Theory}

\section{Introduction}
This chapter presents a very basic introduction to Density Functional
Theory (DFT). DFT is a remarkable theory that currently has wide use
in quantum chemistry, as it has computational requirements comparable to
Hartree-Fock (HF) techniques, but accuracy comparable to more exact
techniques such as MP2.

DFT starts from very different assumptions than does HF, but winds up
with a set of equations (the Kohn-Sham equations) that are extremely
similar to the Fock equations. We will therefore introduce the DFT
one-particle Hamiltonian as a modification of the HF Hamiltonian. This
will allow us to understand most of the salient points about DFT, but
we should bear in mind that we are skipping over some very important
distinctions. 

\section{Review of the Hartre-Fock One-Particle Hamiltonian}
The Fock equations are the one-particle Hamiltonian we solve in HF
theory. For a closed-shell system, the same equation,

\begin{equation}
	F^c = h + 2J^c - K^c,
\label{Fock}
\end{equation}

\noindent can be solved for every particle in the system. Here $h$ is
the part of the Hamiltonian that describes one-electron interactions,
such as the kinetic energy of an electron, or the attraction between
an electron and a nucleus in the molecule. The term $J^c$, called the
\emph{Coulomb term}, describes
the electrostatic repulsion between the electrons in the system. The
term $K^c$, called the \emph{exchange term}, is an artifact of the
antisymmetry requirement that the Pauli exclusion principle puts on
the wave function, and represents the energy change when two electrons
are exchanged.

Recall to get the Fock operator in (\ref{Fock}) we had to make the
\emph{self-consistent field} approximation: electrons no longer see
other individual electrons, but rather the average field of all of the
other electrons. In reality, however, electrons \emph{do} see each
other. The terms that the self-consistent field approximation leaves
out are collectively referred to as \emph{electron correlation}. 

As an example of electron correlation, consider the He atom, which has
two electrons. The self-consistent field approximation says that there
is as much chance of the two electrons being right next to each other
as there is for them to be on opposite sides of the molecule. Clearly,
since electrons are negatively charged and repel each other, this is
not the case.

\section{The Kohn-Sham Hamiltonian as an extension of the Fock
equations}
So then what does the true Hamiltonian look like, if we know that the
Fock equations are an over-simplification. Perhaps we can take it to
look something like

\begin{equation}
F = h + 2J - K + V_{corr}.
\end{equation}

\noindent That is, the Hamiltonian is roughly the same as the Fock
equation, but has a correction term in it to add in electron
correlation effects.

Is it even plausible that such a statement might work? In two landmark
papers, Hohenberg and Kohn proved that any property
of the ground state wave function can also be derived from the ground
state density, and Kohn and Sham derived a set of
one-particle self-consistent equations for exchange and correlation
effects based on such a \emph{density functional} treatment.

Based on such a treatment, the one-particle equation looks like 

\begin{equation}
F = h + v_J[\rho] + v_X[\rho] + v_C[\rho].
\label{Fks}
\end{equation}

\noindent Here the HF Coulomb and exchange operators are now written
as functionals of the ground state electron density $\rho$, and there
is an additional correlation term that is also a function of $\rho$.

\section{The Local Density Approximation (LDA)}
But what form are the functionals in equation (\ref{Fks}) to take? The
initial guess that people made were that these functionals were
actually functionals of the \emph{local density}. That is, the total
correlation energy is a sum over lots of points of some function of
the density at that point:

\begin{equation}
	E_C = \sum_i F(\rho(r_i)).
\end{equation}

\noindent There is a compelling elegance to such an approximation. We
saw earlier that one of the problems with excluding electron
correlation was that there was just as much chance that two electrons
in the same orbital were right on top of one another as far
apart. Perhaps we could penalize regions in the molecule that had too
high density to keep this from happening.

In 1980 Ceperley and Alder performed Quantum Monte
Carlo calculations on the homogeneous electron gas at different
densities, and allowed very accurate correlation potentials to be
derived. The resulting LDA
potentials have been found to work well for geometries, particularly
in solids, but do not work as well for the energetics of finite
molecules. 

\section{The Generalized Gradient Approximation (GGA)}
Of course, molecules are not homogeneous electron gases, they contain
very inhomogeneous distributions of electrons, peaked near the
nucleus. In fact, it is the particular nature of these inhomogeneities
that gives rise to chemical bonding, so if we want chemical accuracy
($\approx 5$ kcal/mol) from our method, we need to make sure we can 
treat inhomogeneities accurately.

So perhaps, rather than making functionals dependent upon only the
local density, we could allow them to be a function of the local
density, and the gradient of the local density:

\begin{equation}
	E_C = \sum_i F(\rho(r_i),\nabla\rho(r_i)).
\end{equation}

\noindent Such an approximation, due mostly to Perdew 
and Becke allows the functional to
more accurately describe inhomogeneities in the electron density. This
means that these functionals, which go by names BLYP, PW91, and PBE,
among others, can begin to describe molecules with chemical accuracy.

\section{Exact Exchange}
One of the things that this note has been glossing over is that the
exchange operator $v_X$ is itself a function of the density, and, like
the correlation functional, is a fit to homogeneous and inhomogeneous
results. However, we already know that under some boundary conditions,
it should look pretty much like the HF exchange operator
$K$. Following these arguments, Becke experimented with
\emph{hybrid functionals} that have some of the traditional GGA
exchange functional $v_X$ in them, and some amount of the exact
(i.e. HF) exchange operator $K$. By trial and error he found that
$\approx30\%$ was optimal for a variety of molecules. The most
commonly used functional is the B3LYP functional, which consists of
Becke's three-term hybrid exchange functional, and the LYP correlation
functional. 

\section{The Kohn Sham Equations}

\section{Practical Issues}

\subsection{The Poisson Equation}

\subsection{Atomic Grids}

\section{Performance of a Family of Density Functionals}
It is sometimes useful to understand how different density functionals
perform over a wide range of molecules. Pople and coworkers have done
two excellent studies \cite{Johnson93,Curtiss96}. Table
\ref{table:popdft} summarizes information from this study.

\begin{table}
\caption{Summary of data from \cite{Curtiss96} comparing the
performance of different density functionals for computing enthalpies
of formation of a wide range of molecules. Average absolute deviations
from experimental results are reported in kcal/mol.}
\label{table:popdft}
\begin{tabular}{lrrrrrrrr}\hline\hline
       &              &              & Substituted &   & Inorganic & \\
Method & Non-hydrogen & Hydrocarbons & Hydrocarbons & Radicals &
	Hydrides & Overall \\ \hline
SVWN & 73.58 & 133.71 & 124.41 & 54.56 & 33.65 & 90.88\\
BLYP & 10.30 & 8.09 & 6.10 & 5.09 & 3.13 & 7.09 \\
BPW91 & 12.25 & 4.85 & 7.99 & 6.48 & 4.21 & 7.85 \\
BP86 & 16.61 & 25.82 & 26.80 & 15.76 & 8.16 & 20.19 \\
B3LYP & 5.35 & 2.76 & 2.10 & 2.98 & 1.84 & 3.11 \\
B3PW91 & 5.14 & 3.96 & 2.77 & 3.21 & 1.99 & 3.51 \\
B3P96 & 7.80 & 30.81 & 25.49 & 13.53 & 7.86 & 17.97 \\
\hline\hline
\end{tabular}
\end{table}

One should not take such averages too seriously, but in general terms,
these studies by Pople confirm results found by most other workers in
the field. With quality (e.g. DZP) basis sets, one can normally
anticipate 5-10 kcal/mol accuracy from GGA functionals such as BLYP or
PW91, and 3-5 kcal/mol accuracy from Hybrid functionals such as
B3LYP. One must be very cautious with the energies from LDA
calculations such as SVWN, but the structures are very often reliable.


\section{Suggestions for Further Reading}
Walter Kohn's two landmark papers \cite{Hohenberg64,Kohn65} are both
examples of that rare beast: a paper that is both an intellectually
signficant breakthrough and that is easy to read. Parr and Yang's book
\emph{Density Functional Theory of Atoms and Molecules} presents a
very thorough treatment of the entire field, but only covers through
the development of LDA. Perdew's \cite{Perdew86} and Becke's
\cite{Becke86, Becke88a, Becke88b} papers present good descriptions of
GGA, and Becke's 1993 paper \cite{Becke93} presents a good description
of hybrid functionals.
