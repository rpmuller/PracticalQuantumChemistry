\chapter{Open-Shell Hartree-Fock Theory}

\section{Introduction}
Chapter \ref{chap:simplehf} described Hartree-Fock theory for
closed-shell wave functions, where every electron was spin paired with
another in a doubly-occupied orbital. Most molecules that we encounter
in the everyday life (water, ethanol, and acetic acid, to mention a
few that are probably in your kitchen) are closed-shell
molecules. However, some molecules (most notably oxygen), and many
important fragments of molecules, contain one or more unpaired
electrons residing in singly-occupied orbitals. One way to solve these
system is to treat the up-spin and down-spin electrons as if they were
in two separate universes; this technique is called \emph{unrestricted
Hartree-Fock} (UHF) and is described in section \ref{sec:uhf}. UHF
techniques have the shortcoming that the orbitals for corresponding
up- and down-spins aren't required to be equivalent. This is a
phenomenon called \emph{spin contamination} is typically incorrect,
although it can lead to more accurate descriptions of wave functions
as chemical bonds dissociate. The technique that requires the up- and
down-spin electrons to be in equivalent orbitals is called
\emph{restricted open-shell Hartree-Fock} (ROHF), and is described in
section
\ref{sec:rohf}. 


\section{Unrestricted Open-Shell Hartree-Fock Theory}
\label{sec:uhf}
A simple way of treating systems with unpaired electrons is
\emph{unrestricted Hartree-Fock} (UHF) theory. UHF solves different
one-particle equations for the up- and the down-spin electrons. 

Consider a section with $N_\alpha$ electrons with $\alpha$ spin, and
$N_\beta$ electrons with $\beta$ spin. The UHF analogue of the Fock
equations are called the \emph{Pople-Nesbet} equations. These are

\begin{equation}
F^\alpha = h + \sum_i^{N_\alpha} \left(J_i^\alpha-K_i^\alpha\right)
             + \sum_i^{N_\beta} J_i^\beta,
\label{eq:pnalpha}
\end{equation}
\begin{equation}
F^\beta = h + \sum_i^{N_\beta} \left(J_i^\beta-K_i^\beta\right)
             + \sum_i^{N_\alpha} J_i^\alpha.
\label{eq:pnbeta}
\end{equation}

\noindent That is, each $\alpha$ spin electron (equation
(\ref{eq:pnalpha})) has a Coulomb and an exchange interaction with
each of the other $\alpha$ electrons, and only a Coulomb
interaction with the $\beta$ electrons; each $\beta$ spin electron
(equation (\ref{eq:pnbeta})) has a Coulomb and an exchange interaction
with each of the other $\beta$ electrons, and only a Coulomb
interaction with the $\alpha$ electrons. Recall that exchange
interaction only occurs between orbitals with the same spin.

UHF calculations do a good job of describing unpaired spins. One
shortcoming to the approach is that there is nothing to require that
orbital $\phi_i^\alpha$ has any similarity to orbital
$\phi_i^\beta$. This contradicts our normal chemical intuition about
these systems, and it also means that the wave function is no longer a
pure spin state. This effect is known as \emph{spin contamination}.

\section{Restricted Open-Shell Hartree-Fock Theory}
\label{sec:rohf}
An alternative approach is to \emph{restrict} the $\alpha$ orbitals to
be the same as the $\beta$ orbitals, a technique known as
\emph{restricted open-shell Hartree-Fock} (ROHF). For
a molecule that consists of $N_c$ doubly-occupied core orbitals and
$N_o$ singly-occupied, high-spin coupled open-shell orbitals, the wave
function is given by 

\begin{equation}
  \Psi = |\phi_1\bar{\phi_1}\cdots\phi_{N_c}\bar{\phi_{N_c}}
	\phi_{N_c+1}\cdots\phi_{N_c+N_o}\rangle
\label{eq:wfngen}
\end{equation}

\noindent The energy of this wave function is given by

\begin{equation}
  E_{el} = \sum_i^{N_c+N_o}2f_ih_{ii}+ \sum_{ij}^{N_c+N_o}
	(a_{ij}J_{ij}+b_{ij}K_{ij})
\label{eq:eos}
\end{equation}

\noindent where $f_i$ is the occupation for orbital $i$

\begin{equation}
	f_i = \left\{\begin{array}{ll}
		1 & \mbox{$\phi_i$ is doubly-occupied} \\
		1/2 &  \mbox{$\phi_i$ is singly-occupied}
	\end{array}\right.
\label{eq:fdef}
\end{equation}

\noindent and

\begin{equation}
	a_{ij} = 2f_if_j
\label{eq:aij}
\end{equation}

\begin{equation}
	b_{ij} = -f_if_j 
\label{eq:bij}
\end{equation}

\noindent with the added condition that $b_{ij} = -1/2$ if $i$ and $j$
are both open-shell orbitals. Note that \ref{eq:eos} reduces to
\ref{eq:eel} when all orbitals are doubly occupied.

Because an open-shell wave function of the form of \ref{eq:wfngen}
requires two sets of $f_i$ coefficients ($f_i = 1$ and $f_i = 1/2$),
the wave function is said to have two shells. Because $(N_o+1)$ Fock 
operators are required (see next paragraph), the wave function is said
to have $(N_o+1)$ Hamiltonians. 

A special case of restricted open-shell Hartree-Fock is the
\emph{open-shell singlet} wave function, where the highest two
orbitals are both singly-occupied and singlet paired. Denoting these
two orbitals $\phi_i$ and $\phi_j$, if they have a triplet pairing
\begin{equation}
\phi_i\phi_j\alpha\alpha, 
\end{equation}
the component of the energy corresponding to these two orbitals would
be 
\begin{equation}
 E_ij = J_{ij} - K_{ij}
\end{equation}
which is consistent with the definitions in equations (\ref{eq:fdef})
-- (\ref{eq:bij}). However, if the two orbitals have a singlet pairing
\begin{equation}
(\phi_i\phi_j+\phi_j\phi_i)(\alpha\beta-\beta\alpha)
\end{equation}
the component of the energy corresponding to these two orbitals is now
\begin{equation}
 E_ij = J_{ij} + K_{ij},
\end{equation}
corresponding to $a_{ij}=+1/2$ and $b_{ij}=+1/2$.

Open-shell singlet wave functions correspond to excited states within
the restricted Hartree-Fock approximation, and are a good way of
describing excited states without having to perform an expensive
configuration interaction calculation.

The procedure for optimizing the orbitals is
slightly more complicated with the open-shell wave
function in \ref{eq:wfngen} than it was for the closed shell
wave function in \ref{eq:prodwf}. For optimizing the occupied
orbitals with the unoccupied orbitals $(N_o+1)$ Fock
operators are now required. For the core orbitals, $F^c$,
given by 

\begin{equation}
  F_{\mu\nu}^c = f_ch + \sum_k^{N_{ham}}(a_{ck}J_{\mu\nu}^k
	+b_{ck}K_{\mu\nu}^k)
\label{eq:fc}
\end{equation}

\noindent is formed, and for each open-shell orbital 

\begin{equation}
  F_{\mu\nu}^i = f_ih + \sum_k^{N_{ham}}(a_{ik}J_{\mu\nu}^k
	+b_{ik}K_{\mu\nu}^k)
\end{equation}

\noindent is formed. Note that the summation here is over $N_{ham}$,
the number of Hamiltonians, equal to $N_o+1$. One density matrix,
$D_c$, is formed for all the core orbitals, and density matrices $D_i$
are formed for each of the open-shell orbitals. The Coulomb and
exchange operators associated with each of these density matrices are
given by

\begin{equation}
  J_{\mu\nu}^k = \sum_{\sigma\eta}^{N_{bf}}D_{\sigma\eta}^k
	(\mu\nu|\sigma\eta)
\label{eq:jkop}
\end{equation}

\noindent and

\begin{equation}
  K_{\mu\nu}^k = \sum_{\sigma\eta}^{N_{bf}}D_{\sigma\eta}^k
	(\mu\sigma|\nu\eta)
\label{eq:kkop}
\end{equation}

\noindent Forming the matrices in this fashion saves considerable
effort because only one Coulomb and one exchange matrix are needed for
all of the $N_c$ core orbitals 

Once the $N_{ham}$ Fock operators are formed in the basis function
space, they are again transformed into molecular orbital
space by multiplying by the appropriate transformation
coefficients 

\begin{equation}
  F_{ij}^k = \sum_{\mu\nu}^{N_{bf}}c_{\mu i}c_{\nu j}F_{\mu\nu}^k
\label{eq:fij}
\end{equation}

\noindent For the core Fock operator $F^c$, $i$ and $j$ range over all
core orbitals and all unoccupied orbitals; for the open shell Fock
operator $F^k$, $i$ and $j$ range over open-shell orbital $k$ and all
unoccupied orbitals. Once each Fock operator is transformed to the
molecular orbital basis it is diagonalized; the eigenvectors yield the
linear combination of orbitals required for the next more optimized
set of orbitals, and the eigenvalues yield the orbital energies.

Unlike the closed-shell example given earlier, mixing occupied
orbitals that have different $f$, $a$, and $b$ coefficients can change
the energy. Starting from the energy expression \ref{eq:eos} and taking the
pairwise mixing of orbitals $\phi_i$ and $\phi_j$ 

\begin{equation}
  \phi_i^\prime = \frac{\phi_i + \delta_{ij}\phi_j}
	{\sqrt{1+\delta_{ij}^2}}
\end{equation}

\noindent and

\begin{equation}
  \phi_j^\prime = \frac{\phi_j + \delta_{ij}\phi_i}
	{\sqrt{1+\delta_{ij}^2}}
\end{equation}

\noindent to preserve orthonormality, and expanding through second
order in $\delta_{ij}$ gives the equation for the change in energy

\begin{equation}
  \Delta E_{ij}(1+\delta_{ij}^2) = 2\delta_{ij}A_{ij}+
	\delta_{ij}^2B_{ij}
\label{eq:emod}
\end{equation}

\noindent where 

\begin{equation}
  A_{ij} = \langle i|F^j-F^i| j \rangle
\end{equation}

\begin{equation}
  B_{ij} = \langle i|F^j-F^i| i \rangle - \langle j|F^j-F^i| j \rangle
	+ \gamma_{ij}
\end{equation}

\begin{equation}
  \gamma_{ij} = 2(a_{ii}+a_{jj}-2a_{ij})K_{ij}
		+ (b_{ii}+b_{jj}-2b_{ij})(J_{ij}+K_{ij})
\label{eq:gamma}
\end{equation}

\noindent Requiring the energy change be stationary 
with respect to $\delta_{ij}$ gives

\begin{equation}
  \delta_{ij} = -\frac{A_{ij}}{B_{ij}}
\end{equation}

To preserve orbital orthonormality, $\delta_{ji} =
-\delta_{ij}$. One way of making these variations simultaneously is to
form the rotation matrix $\Delta$, the anti-symmetric matrix with zero
diagonal defined by

\begin{equation}
  \Delta = \left[\begin{array}{cc}
	0 & \frac{A_{ij}}{B_{ij}} \\
	-\frac{A_{ij}}{B_{ij}} & 0
	\end{array}\right]
\end{equation}

\noindent The new set of orbitals $\{\phi^{New}\}$ are obtained from
the old set of orbitals $\{\phi^{Old}\}$ via the transformation

\begin{equation}
  [\phi^{New}] = \exp(\Delta)[\phi^{Old}]
\label{eq:orbup}
\end{equation}

This method computes the optimal mixing of the occupied orbitals with
respect to each other while preserving orbital orthonormality. 

It should be noted that if two orbitals have the same $f$, $a$, and
$b$ coefficients, as defined by \ref{eq:fdef} and ref{eq:abdef},
$A_{ij} = 0$, and, consequently, no mixing is done between those
orbitals. Thus, when the wave function only
consists of closed-shell orbitals no occupied-occupied mixing is done,
and only occupied-unoccupied mixing is performed via diagonalization
of the Fock operator $F^c$. The occupied-occupied mixing would have no
effect because the numerator of the rotation matrix $D$ would be zero,
and the orbitals would not change.

In a given iteration, first the optimal
occupied-occupied mixing is computed via equation \ref{eq:orbup}, and
the occupied-virtual mixing is computed by forming and diagonalizing
the Fock operators for each Hamiltonian, via equations
\ref{eq:fc}--\ref{eq:fij}. The iterations continue until the orbitals
stop changing, at which point the wave function is said to be
\emph{converged}.

\section{Suggestions for Further Reading}
Szabo and Ostlund \cite{Szabo82} provide a brief treatment of UHF
theory in Chapter 3. Goddard and Bobrowicz's article
\emph{The Self-Consistent Field Equations for Generalized Valence Bond
and Open-Shell Hartree-Fock Wave Functions}\cite{Bobrowicz77},
presents a very detailed derivation of ROHF theory.
