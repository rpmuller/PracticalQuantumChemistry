\chapter{Gaussian Basis Sets}
\label{Basis_Chapter}

\section{Introduction}
This chapter presents an introduction to the Gaussian basis sets used
in electronic structure calculations. Fundamentally, we want to be
able to adjust the shape of the orbitals to find the ones that yield
the most stable energy. To adjust the shape, we use the
\emph{linear combination of atomic orbitals} (LCAO) approach, where we
expand an orbital $\phi_i$ in a set of 
\emph{basis functions} $\{\chi_\mu\}$
\begin{equation}
	\phi_i = \sum_\mu^{nbf} c_{\mu i} \chi_\mu.
\end{equation}
By adjusting the coefficients $c_{\mu i}$ we can change the shape of
the orbital $\phi_i$, which transforms the problem of finding the best
orbital shape to a linear algebra problem.

\section{Slater Functions}
The first basis sets used were of the form 
\begin{equation}
	\chi(r) = Y_{lm}\exp\{-\zeta r\}.
\end{equation}
These were used because they have the same form as atomic
orbitals. However, integrals between these functions proved
difficult, and they have largely been abandoned in favor of Gaussian
basis functions. Nonetheless, there are still two important advantages
that Slater functions have over Gaussians. As $r\rightarrow 0$ Slater
functions properly form a \emph{cusp} (that is, they have finite
slope), whereas Gaussians go to zero with zero slope. Moreover, as
$r\rightarrow \infty$, Gaussian functions fall off too quickly, as
compated to the Slater functions.

\section{Gaussian Basis Sets}
In 1950 Boys suggested that Gaussian functions
\begin{equation}
	\chi(r) = x^iy^jz^k\exp\{-\alpha r^2\}
\end{equation}
might be a good type of basis function for quantum chemistry, because
of the fact that the product of two Gaussians is also a Gaussian. When
the exponents $(i,j,k)$ are zero,
\begin{equation}
	\exp\{-\alpha_Ar_A^2\}\exp\{-\alpha_Br_B^2\} = 
	\exp\left\{\frac{-\alpha_A\alpha_Br_{AB}^2}{\gamma}\right\}
	\exp\{-\gamma r_p^2\},
\end{equation}

\noindent where

\begin{equation}
	\gamma = \alpha_A+\alpha_B,
\end{equation}

\begin{equation}
	r_p = \frac{\alpha_Ar_A+\alpha_Br_B}{\gamma}.
\end{equation}

\noindent When the exponents $(i,j,k)$ are nonzero, we can expand in a
binomial series

\begin{equation}
	x_A^{i1}x_B^{i2} = (x_p+x_{pA})^{i1}(x_p+x_{pB})^{i2}
		= \sum_qf_q(i1,i2,x_{pA},x_{pB})x_p^q,
\end{equation}

\noindent and we can once again express the product as a sum of
Gaussians. The point to remember was that Gaussians were proposed and
adopted because they were numerically easier to deal with, rather than
being more accurate.

\subsection{Contracted Basis Sets}

Because Gaussian basis functions do not have the proper cusp and tail
behavior that Slater functions do, we need to \emph{contract} several
of them together to form a single basis function. In the jargon, the
functions that are contracted together are each called \emph{primitive
basis functions}, and the resulting function is referred to as a
\emph{contracted basis function}. When quantum chemists refer to a
\emph{basis function} they typically mean a \emph{contracted basis
function}. 

% \subsection{Split Valence Basis Sets}

\subsection{Naming Gaussian Basis Sets}
The naming of commonly used Gaussian basis sets can also be
confusing. This section will attempt to briefly categorize some major
families of basis sets.

The smallest basis set is (not surprisingly) called a \emph{minimal
basis} (MB) description. This basis set has one basis function per
occupied atomic orbital on the atom. Thus, H would have one basis
function for the $1s$ orbital, and C would have 5 basis functions for
the $1s$, $2s$, $2p_x$, $2p_y$, $2p_z$ orbitals, in a minimal basis 
description. The most commonly used MB basis set has the name STO-3G,
to signify that three contracted Gaussians are used to replace one
Slater type orbital.

The problem with the minimal basis description is that it doesn't
allow the atoms to change their shape very much. If H only has one
function, it doesn't have any degrees of freedom to adjust to adapt to
a different bonding situation. 

Thus, we can augment our basis sets with an additional basis function
per occupied atomic orbital. These basis sets are called \emph{double
zeta} (DZ) basis sets. In practice most DZ basis sets are in fact
really \emph{valence double zeta} (VDZ) basis sets, because it is only
necessary to add additional functions for the valence orbitals, and
not the core orbitals, since only the valence orbitals are involved in
chemical bonding. A VDZ description of H has 2 basis functions ($1s$
and $2s$), and a VDZ description of C has 9 basis function ($1s$,
$2s$, $2p_x$, $2p_y$, $2p_z$, $3s$, $3p_x$, $3p_y$, $3p_z$).  Commonly
used VDZ basis sets have the names 3-21G and 6-31G; the names signify
the number and type of basis functions used in the contraction.

But even a VDZ description doesn't have as much freedom as we would
like. Molecules often polarize when near a large charge or in the
presence of an external field. VDZ basis sets let the functions get
larger or smaller, but don't let them polarize. So we add
\emph{polarization functions}, which are functions with one higher
angular momentum than the highest occupied basis function. Thus, for
H, which has occupied $1s$ functions, we would add $2p$ functions, and
for C, which has occupied $2p$ functions, we would add $3d$
functions. This description is now called \emph{double zeta plus
polarization} (DZP). H now requires 5 basis functions ($1s$, $2s$,
$2p_x$, $2p_y$, $2p_z$ orbitals), and C requires 15 functions ($1s$,
$2s$, $2p_x$, $2p_y$, $2p_z$, $3s$, $3p_x$, $3p_y$, $3p_z$, $3d_{xx}$,
$3d_{yy}$, $3d_{zz}$, $3d_{xy}$, $3d_{yz}$, $3d_{xz}$). Commonly used
DZP basis sets are the 6-31G** basis set and the cc-pvDZ basis
set. Sometimes, a 6-31G* basis set is used; here the single *
indicates that polarization functions are included on the heavy atoms
(everything except H) but not on H itself. The 6-31G** basis is one of
the most commonly used basis sets, and is a good default basis set to
use when studying a new molecule.

Finally, even DZP basis sets have a difficult time treating negative
ions, which typically are in orbitals much more diffuse than the
valence orbitals. Diffuse orbitals are also often useful in describing
excited states of molecules. For these cases, we augment our basis
sets with a set of \emph{diffuse} basis functions. These basis sets
are typically denoted using a + suffix, to denote diffuse functions
on the heavy atoms, or a ++ suffix to denote diffuse functions on all
atoms. 

Note that when $d$-functions are included there are six functions listed
($d_{xx}$, $d_{yy}$, $d_{zz}$, $d_{xy}$, $d_{yz}$, $d_{xz}$), rather
than the 5 that most students of chemistry are familiar with
($d_{xx-yy}$, $d_{zz}$, $d_{xy}$, $d_{yz}$, $d_{xz}$). The six
functions are the simple Cartesian forms of the basis functions, and
are the easiest way to input the basis functions. However, note that
with the six functions listed, it is possible to make a linear
combination, $d_{xx} + d_{yy} + d_{zz}$, that actually has $s$-like
angular momentum. Keeping all six functions leads to a basis set that
has extra description of $s$-type angular momentum. Such a description
isn't wrong, but might lead to an unbalanced description: the $s$-type
angular momentum has a more accurate description than the
$d$-type. Many people believe that such an unbalanced description is
misleading, and thus remove the $s$-combination of the $d$-basis
functions. Many people do not remove this combination. 

Table \ref{table:bsets} summarizes basis function information.

\begin{table}
\caption{Different Levels of Basis Functions. Shown is the level, the
number of basis functions used for H and C, and some commonly used
examples.}
\label{table:bsets}
\begin{tabular}{llll}\hline\hline
Level & $N_{bf}$[H] & $N_{bf}$[C] & Examples \\  \hline
Minimal Basis (MB) & 1 & 5 & STO-3G \\
Double $\zeta$ (DZ) & 2 & 9 & 3-21G, 4-31G, 6-31G \\
Double $\zeta$ + polarization (DZP) & 5 & 15 & 6-31G**, cc-pVDZ\\ \hline\hline
\end{tabular}
\end{table}

\section{Suggestions for Further Reading}
Boys's 1950 article \cite{Boys50} has a very good treatment of
Gaussian basis functions and the reasons why they were preferred to
Slater functions. 