\chapter{Thermochemistry from Quantum Chemistry Calculations}
\label{chap-thermochem}

\section{Introduction}
We can obtain a great deal of information from quantum chemistry
calculation, and this section will discuss the different types of
data and how they are obtained from quantum chemistry.

\section{Frequencies and Thermal Properties}

When we discussed geometry optimization we briefly mentioned the
$Hessian matrix$, which is the matrix that contains the
second-derivative of the energy with respect to nuclear
displacements. From this matrix we may obtain both the \emph{normal
modes of vibration} and the \emph{frequencies} associated with these
modes. One can use these frequencies, for example, in analyzing
infra-red spectra, and they are also often valuable in fitting quantum
chemical results to force fields. 

We may also use the frequencies to estimate the thermal properties of
molecules. Quantum mechanics tells us that a molecule is never at
rest, even at absolute zero; the molecule typically has some
\emph{zero-point energy} in its vibrational modes. The frequencies may
be used to estimate this energy. Moreover, the frequencies may be used
to produce heat capacities, entropies, enthalpies, and free energies
of the molecule at different temperatures. This information can be
particularly useful in comparing to experimental results; the
zero-point energy corrected free energy at the experimental
temperature is generally the proper quantity to consider rather than
the total energy in comparing the results of quantum chemistry
caluclations to experiment.

Finally, the frequencies and vibrational modes are often
particularly useful in analyzing transition states. A proper
transition state should have one imaginary frequency. The mode
corresponding to that frequency should connect the reactant and
product geometries. That is, the vibrational mode should correspond to
motion along the reaction path that connects the reactants and
products. If the vibrational mode does not exhibit this behavior, it
is often a sign that something is wrong with the transition state.

\section{Reaction Energetics}

The ability to compute thermal properties as described in the previous
section now makes it possible to get accurate approximations of
chemical reactions from quantum chemistry. Suppose we are interested
in the $S_N2$ bimolecular nucleophilic reaction Cl- + CH$_3$Br
$\rightarrow$ CH$_3$Cl + Br-. If we were interested in the gas-phase
reaction we would compute ground state structures for Cl-, Br-,
CH$_3$Br, and CH$_3$Cl, and a $C_3v$ transition state for
Cl-CH$_3$-Br. Each of these structures should have optimized
geometries, zero-point energies, and thermochemical data (heat
capacity, entropy, enthalpy, and free energy) for the temperatures of
interest. We are then able to compute enthalpy and free energy curves
for the reactions, and, if desired, use transition state theory to
compute rate constants for the reaction kinetics. The process may also
be repeated for solution phase reactions.

\section{Heats of Formation}
It is often important to compute \emph{heats of formation}, $\Delta
H_f$, the enthalpy required to form the molecule from standard forms
of the elements. Suppose we want to compute the heat of formation of
methane, CH$_4$. The standard form of hydrogen is H$_2$ gas, which is
easy to compute using quantum chemistry techniques, but the standard
form of carbon is diamond, which is an infinite solid, and which is
not easy to compute using the techniques we have discussed so far. A
further complication is that very often the 5 kcal/mol accuracy in
computing the heat of formation is not accurate enough for the desired
application. In this case the best way to obtain quality heats of
formation is to compute a range of molecules whose experimental heats
of formation are known (the NIST WebBook at
http://webbook.nist.gov/chemistry/ is an excellent source for
experimental heats of formation), and to then determine by
least-squares fitting a set of correction factors to the raw heats of
formation. This experimentally corrected approach generally leads to
RMS errors of $< 3$ kcal/mol, which is accurate enough for many
applications. 

\subsection{Simple Fits to Ensembles of Data}

\subsection{G2 and G3 Theory}

\section{Summary}

