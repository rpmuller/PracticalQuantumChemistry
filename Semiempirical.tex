\chapter{Semiempirical Quantum Chemistry Calculations}

\section{Introduction}
Semiempirical techniques offer good accuracy at very modest
computational cost for a number of different systems. These techniques
start with the Fock equations (\ref{eq:fock}), but rather than attempt
to solve it, first eliminates most of the small elements, and then
approximates the rest of the terms with experimental values. 

This note covers the theoretical underpinnings of the different
semiempirical techniques for computing
the electronic structure of molecules \cite{Pople65, Pople65a,
Pople66, Pople70}.

This note is limited to closed-shell molecules. I take some liberties
during the derivations. In particular, I mix parts of Pople's original
theory with later developments that I believe to be improvements (in
particular the Mataga-Nishimoto parameterization of the two-electron
integrals). Moreover, my CNDO development follows the spirit of CNDO/2
rather than CNDO/1. I believe that these departures from Pople's
original theory make the resulting theory easier to understand and
faster to implement.

\section{The Self Consistent Field Equations}
We start by assuming that the wave function of a closed-shell molecule
may be written as a product of doubly-occupied orbitals

\begin{equation}
	\Psi = \prod_i^{N_{occ}}\phi_i^2(\alpha\beta-\beta\alpha).
\end{equation}

\noindent We also assume that these orbitals may be accurately
expanded as a linear combination of atomic-like orbitals

\begin{equation}
	\phi_i = \sum_{\mu}^{N_{bf}} c_{\mu i}\chi_\mu.
\end{equation}

\noindent Variational treatment of the orbital coefficients $c_{\mu
i}$ leads to the Fock equations

\begin{equation}
	\sum_\nu F_{\mu\nu}c_{\mu i} = \sum_\nu S_{\mu\nu}c_{\nu i}
		\epsilon_i.
\end{equation}

\noindent Here $F_{\mu\nu}$ is the Fock matrix, whose elements are
given by 

\begin{equation}
	F_{\mu\nu} = t_{\mu\nu} + v^N_{\mu\nu} 
		+ \sum_{\sigma\eta}P_{\sigma\eta}\left[
			(\mu\nu|\sigma\eta)-\frac{1}{2}(\mu\sigma|\nu\eta)
			\right],
\label{fock}
\end{equation}

\noindent where $t_{\mu\nu}$ is the kinetic energy operator, $v^N$ is
the nuclear attraction operator, and

\begin{equation}
	(\mu\nu|\sigma\eta) = \int\int \frac
		{\chi_\mu^*(1)\chi_\nu(1)\chi_\sigma^*(2)\chi_\eta(2)}
		{r_{12}}d^3r_1d^3r_2
\label{two-e}
\end{equation}

\noindent is a two-electron integral.

We will ignore the finer points of computing the terms in equations
(\ref{fock}) and (\ref{two-e}), as the idea behind semiempirical
techniques is to approximate the most important of these terms and to
exclude the rest.

\section{CNDO Approximations}
CNDO makes several simplifying approximations, which will be discussed
in this section.

The fundamental assumption (from which the name of the technique
derives) is that the differential overlap of atomic orbitals on the
same atom is neglected. The first aspect of this is simplification of
the overlap matrix, such that

\begin{equation}
	S_{\mu\nu} = \delta_{\mu\nu}.
\label{ndo-overlap}
\end{equation}

\noindent For the atomic orbitals for which the theory was
developed, the overlap between different atomic orbitals on the same
atom is already zero. Consequently, in that basis, the only elements
excluded are between atomic orbitals on different atoms.

The next approximation simplifies the two electron integrals:

\begin{equation}
	(\mu\nu|\sigma\eta) 
		= (\mu\mu|\sigma\sigma)\delta_{\mu\nu}\delta_{\sigma\eta}
		= \gamma_{\mu\sigma}.
\end{equation}

\noindent The difficulty with this level of approximation is that the
integral is no longer rotationally invarient. Consequently, these
integrals are normally further simplified to represent $s$-type
integrals on the different atoms, and thus depend only on the two
atoms, $A$ and $B$, on which the functions $\mu$ and $\sigma$ reside.

This integral $\gamma_{\mu\nu} = \gamma_{AB}$ may be further
approximated using the Mataga-Nishimoto approximation
\cite{Mataga57, Zerner80}:

\begin{equation}
	\gamma_{AB} = \frac{f_\gamma}
		{\frac{2f_\gamma}{\gamma_{AA}+\gamma_{BB}} + R_{AB}}.
\label{mataga-nishimoto}
\end{equation}

\noindent The $f_\gamma$ parameter is widely \cite{Zerner80} taken to
be 1.2, and the expressions for $\gamma_{AA}$ are taken from Pariser
and Parr \cite{Pariser53}:

\begin{equation}
	\gamma_{AA} = I_A - A_A
\end{equation}

\noindent where $I_A$ and $A_A$ are the ionization potential and the
electron affinity of atom $A$, respectively.

The next approximations deal with the one-electron terms $t_{\mu\nu}$
and $v^N_{\mu\nu}$. The kinetic energy term is treated as a diagonal
parameter

\begin{equation}
	t_{\mu\nu} = t_{\mu\mu}\delta_{\mu\nu}
\end{equation}

\noindent and the diagonal term is parameterized as

\begin{equation}
	t_{\mu\mu} = -\frac{1}{2}(I_\mu + A_\mu),
\end{equation}

\noindent the average of the ionization potential $I_\mu$ and the
electron affinity $A_\mu$ for function $\mu$.

The nuclear attraction integral $v^N$ 

\begin{equation}
	v^N_{\mu\nu} = \sum_B \int \frac{\chi_\mu \chi_\nu}{r_B}d^3r
\end{equation}

\noindent is simplified by ignoring all integrals unless $\mu$ and
$\nu$ are on the same atom $A$

\begin{equation}
	v^N_{\mu\nu} = \left\{
		\begin{array}{ll}
		\sum_B V_{AB}  & \mbox{if $\mu,\nu \in A$} \\
		0 & \mbox{otherwise}
		\end{array}
		\right. .
\end{equation}

\noindent This may be further approximated via

\begin{equation}
	V_{AB} = Z_B\gamma_{AB}.
\end{equation}

The final approximation concernes $F_{\mu\nu}$, the off-diagonal
elements of the Fock matrix. We ignore all of these elements when
$\mu$ and $\nu$ are on the same atom. Of the remaining elements, we
approximate the one-electron terms via

\begin{equation}
	F_{\mu\nu}^1 = \beta_{AB}^0S_{\mu\nu}
\end{equation}

\noindent and the two-electron terms as

\begin{equation}
	F_{\mu\nu}^2 = -\frac{1}{2}P_{\mu\nu}\gamma_{AB}.
\end{equation}

\noindent Here the resonance term $\beta_{AB}^0 = \frac{1}{2}(\beta_A
+ \beta_B)$, and $\beta_A$ and $\beta_B$ are parameters. We note that
here the $S_{\mu\nu}$ elements represent the true overlap, rather than
the diagonal overlap used in equation (\ref{ndo-overlap}).

Combining all of these approximations together, we are left with the
following form of the Fock matrix:

\begin{equation}
	F_{\mu\mu} = -\frac{1}{2}(I_\mu + A_\mu)
		+ [(P_{AA}-Z_A) - \frac{1}{2}(P_{\mu\mu}-1)]\gamma_{AA}
		+ \sum_{B \neq A} (P_{BB}-Z_B)\gamma_{AB},
\end{equation}

\begin{equation}
	F_{\mu\nu} = \beta_{AB}^0S_{\mu\nu} 
		- \frac{1}{2}P_{\mu\nu}\gamma_{AB}.
\end{equation}

\section{CNDO Parameters}
The most important issue in any NDO technique is that of obtaining
accurate parameters. This section will describe some of the more
common sets of parameters.

Table \ref{table-cndo2} describes the original CNDO/2 parameters from
Pople. 

\begin{table}
\caption{CNDO/2 Parameterization (eV)}
\label{table-cndo2}
\begin{center}
\begin{tabular}{cccc}\\ \hline
Element & $\frac{1}{2}(I_s+A_s)$ & $\frac{1}{2}(I_s+A_s)$ & $\beta_A$
\\ \hline
H 	& 7.176 	&		& 9.0 \\ 
Li 	& 3.106 	& 1.258 	& 9.0 \\
Be 	& 5.946		& 2.563		& 13.0 \\
B 	& 9.594		& 4.001		& 17.0 \\
C 	& 14.051	& 5.572		& 21.0 \\
N 	& 19.316	& 7.275		& 25.0 \\
O 	& 25.390	& 9.111		& 31.0 \\
F 	& 32.272	& 11.080	& 39.0 \\
Na 	& 2.804		& 1.302		& 7.720\\
Mg 	& 5.125		& 2.052		& 9.447\\
Al 	& 7.771		& 2.995		& 11.301\\
Si 	& 10.033	& 4.133		& 13.065\\
P 	& 14.033	& 5.464		& 15.070\\
S 	& 17.650	& 6.989		& 18.150\\
Cl 	& 21.591	& 8.708		& 22.330\\ \hline
\end{tabular}
\end{center}
\end{table}

Table \ref{sichel-params} lists the parameterization of Sichel and
Whitehead (reference \cite{Sichel67}) for CNDO; these parameters also
include $\gamma_{AA}$ terms for the Mataga-Nishimoto equation
(equation (\ref{mataga-nishimoto})).

\begin{table}
\caption{Sichel and Whitehead (reference \cite{Sichel67}) Parameters (eV)}
\label{sichel-params}
\begin{center}
\begin{tabular}{llll}\\ \hline
Element & $-\frac{1}{2}(I_s + A_s)$ & $-\frac{1}{2}(I_p + A_p)$ 
	& $\gamma_{AA}$ \\ \hline
H 	& -13.595	& 		& 12.848 \\
Li 	& -4.999	& -3.673	& 3.458 \\
Be 	& -15.543	& -12.280	& 5.953 \\
B 	& -30.371	& -24.702	& 8.048 \\
C 	& -50.686	& -41.530	& 10.333 \\
N 	& -70.093	& -57.848	& 11.308 \\
O 	& -101.306	& -84.284	& 13.907 \\
F 	& -129.544	& -108.933	& 15.233 \\ \hline
\end{tabular}
\end{center}
\end{table}

%\section{Extended Huckel and Tight Binding}

%\section{Intermediate Neglect of Differential Overlap (INDO)}

%\section{NDDO and Mopac}
