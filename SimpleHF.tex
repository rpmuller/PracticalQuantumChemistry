\chapter{Simple Hartree-Fock Theory}
\label{chap:simplehf}

\section{Introduction}
The last chapter showed how we can construct and solve Hamiltonians
for simple one-dimensional systems to get a feeling for how quantum
mechanical systems work. The current chapter, and the rest of the
book, will focus on a particular type of quantum mechanical system,
the electronic structure problem, the study of which is known as
quantum chemistry.

This chapter introduces \emph{Hartree-Fock} (HF) theory. HF is a
suitable place to start a discussion about practical quantum chemistry
because the equations are relatively simple and yet much profound
chemistry can be described by them. The problem that faces quantum
chemistry is the Schrodinger equation. Typically, Schrodinger
equations of more than a few particles cannot be solved exactly, and a
molecule with $N_A$ nuclei and $N_e$ electrons has $N_p = N_A + N_e$
particles. Fortunately, the \emph{Born-Oppenheimer approximation}
(section \ref{sec:eham}) lets us reduce the number of particles to
only the $N_e$ electrons, and the \emph{self-consistent field
approximation} (section \ref{sec:scf}) reduces the problem to having
to solve for one particle (which is now an \emph{orbital} that holds
two electrons) at a time. The resulting equations are the Fock
equations (section \ref{sec:fock}), whose solutions are expressed as a
linear combination of atomic orbitals (section \ref{sec:basis}). We
conclude with a description of properties that can be obtained from HF
calculations (section \ref{sec:hfprops}).

\section{The Electronic Hamiltonian}
\label{sec:eham}
Electronic structure theory seeks to find solutions to the
non-relativistic time-independent Schrodinger equation 
\begin{equation} 
	H\Psi = E\Psi	
\end{equation}
where $H$ is the Hamiltonian operator for the nuclei and
electrons in a molecule. For these molecular systems, $H$ is given by 
\begin{equation}
  H = -\sum_{i=1}^{N_{el}}t_i
	- \sum_{A=1}^{N_{at}}t_A
	- \sum_{i=1}^{N_{el}}\sum_{A=1}^{N_{at}}\frac{Z_A}{r_{iA}} 
	+ \sum_{i>j}^{N_{el}}\frac{1}{r_{ij}}
	+ \sum_{A=1}^{N_{at}}\sum_{B=1}^{N_{at}}\frac{Z_AZ_B}{R_{AB}}
\end{equation}
for a system with $N_{el}$ electrons and $N_{at}$ nuclei,
where the quantities in $H$ are expressed in atomic units, $t_i$ and
$t_A$ are the kinetic energy of electron $i$ and nuclei $A$,
respectively, $M_A$ is the nuclear mass of atom $A$ in units of
electron mass, $Z_A$ is the charge on nucleus $A$, $r_{iA}$ is the
distance of electron $i$ from nucleus $A$, $r_{ij}$ is the distance
between electrons $i$ and $j$, and $R_{AB}$ is the distance between
nuclei $A$ and $B$.

The molecular Hamiltonian is commonly simplified using the
\emph{Born-Oppenheimer approximation}, which derives from the observation
that nuclei are much heavier than electrons (protons and neutrons are
some 1800 times heavier than an electron, and most nuclei have many
protons and neutrons), and consequently move much more slowly than do
electrons.  To a good approximation one can fix the nuclear
coordinates and consider only the electronic part of the
Hamiltonian. The consequence of this approximation is that the
molecular Hamiltonian $H$ now becomes $H_{el}$, the electronic
Hamiltonian, and is given by 
\begin{equation} 
  H_{el} = -\sum_{i=1}^{N_{el}}t_i
	- \sum_{i=1}^{N_{el}}\sum_{A=1}^{N_{at}}\frac{Z_A}{r_{iA}} 
	+ \sum_{i>j}^{N_{el}}\frac{1}{r_{ij}}
\label{eq:hel}
\end{equation}
For simplicity, the terms in $H_{el}$ involving only one
electron are grouped into a single term $h$,
\begin{equation}
  h = -\sum_{i=1}^{N_{el}} \left(t_i
	+\sum_{A=1}^{N_{at}}\frac{Z_A}{r_{iA}}
	\right)
\end{equation}
and $H_{el}$ is given by
\begin{equation}
  H_{el} = h + \sum_{i>j}^{N_{el}} \frac{1}{r_{ij}}
\end{equation}

The nuclear repulsion energy
\begin{equation}
  E_{nr} = \sum_{A>B}^{N_{at}} \frac{Z_AZ_B}{R_{AB}}
\end{equation}
is constant for a fixed geometry and can be evaluated
separately. $H$ will hereafter refer to only the electronic
Hamiltonian $H_{el}$.

\section{The Molecular Wave Function}
The solutions to $H$ are in the form of a product of molecular
orbitals 
\begin{equation}
  \Psi = \prod_i^{N_{el}} \psi_i.
\end{equation}
The molecular spin orbitals $\psi_i$ are composed of a
spatial function $\phi_i$ and a spin function $\theta_i$. The spatial
orbital $\phi_i$ is a function of the position $r_i$ of electron
$i$. $\phi_i$ describes the spatial distribution of electron $i$ such
that $|\phi_i(r)|^2dr$ is the probability of finding the electron in
the volume element $dr$. This probability for each orbital integrated
over all space must be one, giving the normalization condition
\begin{equation}
  \int\phi_i^*(r)\phi_i(r)dr = 1
\end{equation}
Spatial molecular orbitals can be taken to form an
orthonormal set 
\begin{equation}
  \int\phi_i^*(r)\phi_j(r)dr = \delta_{ij}
\end{equation}

Orbital $\phi_i$ also has a spin component $\theta_i$. The
spin of an electron in orbital $\phi_i$ is described by one of the
orthogonal pair of functions $\alpha$---\emph{spin up}---and
$\beta$---\emph{spin down}. 
Each spatial orbital can accommodate one
electron with $\alpha$-spin, and one electron with $\beta$-spin. Thus,
the simple product wave function has the form 
\begin{equation}
	\phi_1\alpha\phi_1\beta\phi_2\alpha\phi_2\beta\cdots
	\phi_{N_{occ}}\alpha\phi_{N_{occ}}\beta
\label{eq:prodwfn}
\end{equation}
where $N_{occ} = N_{el}/2$. The wave function for an
electron that describes both the spatial and spin components is the
spin orbital $\psi$
\begin{equation}
	\psi_1 = \phi_1\alpha, 
\end{equation}
\begin{equation}
	\psi_2 = \phi_1\beta,  
\end{equation}
and so on.  Spin orbitals are convenient for evaluating many
of the energy expressions in electronic structure theory; Appendix ???
describes techniques for evaluating matrix elements using spin
orbitals.

Because each of the individual orbitals is normalized, the total
probability of finding an electron anywhere in the
wave function is equal to $N_{el}$. 
\begin{equation}
  |\Psi|^2 = \int\Psi^*(1\cdots N_{el})\Psi(1\cdots N_{el})
	dr_1\dots dr_{N_{el}}
\end{equation}

\section{Antisymmetry and the Slater Determinant}
The Pauli exclusion principle states that a wave function
must change sign when the spatial and spin components of any two
electrons are exchanged. 
\begin{equation}  
	\Psi(1,2,\cdots,i,\cdots,j,\cdots,N_{el}) = 
	-\Psi(1,2,\cdots,j,\cdots,i,\cdots,N_{el})
\end{equation}
The Pauli principle derives from the fact that electrons are
indistinguishable particles, so that observable properties of the wave
function cannot change upon exchange of electrons. Because these
observables depend on $|\Psi|^2$, the wave function must either be
symmetric (having the same sign) or anti-symmetric (having opposite
sign) when electrons are exchanged. In practice only anti-symmetric
wave functions are observed.

Because the wave functions must be anti-symmetric, the simple product
wave function form \ref{eq:prodwfn} will not work. A convenient
method of making a simple product wave function
anti-symmetric is to use a Slater determinant. For the
two electron simple product wave function
$\phi_1(1)\alpha(1)\phi_1(2)\beta(2)$, the anti-symmetric wave
function is given by evaluating  
\begin{eqnarray}
  \Psi(1,2) &=& 2^{-1/2} \begin{array}{|cc|}
		\phi_1(1)\alpha(1) & \phi_1(1) \beta(1) \\
		\phi_1(2)\alpha(2) & \phi_1(2) \beta(2) 
	\end{array} \\
     &=& 2^{-1/2}\phi_1(1)\phi_1(2)(\alpha(1)\beta(2) - 
		\beta(1)\alpha(2))
\end{eqnarray}
The generalization of the Slater determinant to an arbitrary
number of particles (and using spin orbitals to signify an arbitrary
spin coupling) is 
\begin{equation}
  \Psi(1,2,\dots,N) = (N!)^{-1/2} \begin{array}{|cccc|}
	\psi_1(1) & \psi_2(1) & \cdots & \psi_N(1) \\
	\psi_1(2) & \psi_2(2) & \cdots & \psi_N(2) \\
        \vdots    & \vdots &   & \vdots \\
	\psi_1(N) & \psi_2(N) & \cdots & \psi_N(N) \\
	\end{array}
\label{eq:slatdet}
\end{equation}
The $(N!)^{-1/2}$ is the normalization condition. For
convenience, two shorthand notations are often used for
\ref{eq:slatdet}. The first
\begin{equation}
	\Psi(1,\dots,N) = (N!)^{1/2}\mathcal{A}
	\left[\psi_1(1)\psi_2(2)\cdots\psi_N(N)\right]
\end{equation}
uses the anti-symmetry operator $\mathcal{A}$ to represent
the determinant and explicitly normalizes the wave function. The
second notation uses Dirac bracket notation
\begin{equation}
  \Psi(1,\dots,N) = |\psi_1(1)\psi_2(2)\cdots \psi_N(N)\rangle
\end{equation}
to represent both the Slater determinant and the
normalization constant $(N!)^{-1/2}$. Both notations use only the
diagonal of the Slater determinant to represent the wave function.

Because Slater determinants are so commonly used to antisymmetrize
wave functions, individual configurations (as shown above) of a wave
function are often referred to as determinants. 

\section{The Self Consistent Field Approximation}
\label{sec:scf}
Even with the restrictions already made, it is not in general possible
to solve \ref{eq:hel} for many-electron wave functions. Consequently,
we use the \emph{self-consistent field} (SCF) approximation to replace
the many-electron Hamiltonian with many one-electron Hamiltonians. In
the SCF approximation, each electron no longer interacts with other
electrons, but with the \emph{average field} produced by the other
electrons. 

The SCF approximation has some dramatic shortcomings. Consider two
electrons in a single orbital. Because the electrons only see each
other's average field, a configuration where the electrons are next to
each other is just as likely as a configuration where the electrons
are on the other side of the molecule. The interactions
excluded by the SCF approximation are known as \emph{electron
correlation}, and subsequent chapters will deal with ways of
correcting for it.

Applying the SCF approximation to the exact electronic Hamiltonian
given by \ref{eq:hel} is commonly known as the Hartree-Fock (HF)
approximation, which reduces the electronic
Hamiltonian to 
\begin{equation}
	H_{HF} = h + v_{HF}	
\end{equation}
where $v_{HF}$ is a two-electron operator representing the
Hartree-Fock field. 

\section{The Hartree-Fock Energy Expression}
Given a wave function with $N$ doubly-occupied
orbitals
\begin{equation}
  \Psi = |\phi_1(1)\alpha(1)\phi_1(2)\beta(2)\cdots
	\phi_N(2N-1)\alpha(2N-1)\phi_N(2N)\beta(2N)\rangle
\label{eq:prodwf}
\end{equation}
we can simplify the notation by writing
\begin{eqnarray}
  \Psi &=& |\phi_1\bar{\phi_1}\cdots\phi_N\bar{\phi_N}\rangle \\
   &=& |1\bar{1}\cdots N\bar{N}\rangle
\end{eqnarray}
where a bar over an orbital signifies spin down, and no bar signifies
spin up, and the order of the orbitals implies electron index. The
energy of $\Psi$ is thus given by  
\begin{eqnarray}
  E &=&  \frac {\langle\Psi|H|\Psi\rangle}{\langle\Psi|\Psi\rangle} \\
    &=&  \frac {\langle 1\bar{1}\cdots N\bar{N}|h + v^{HF}|
	1\bar{1}\cdots N\bar{N} \rangle}
	{\langle 1\bar{1}\cdots N\bar{N}|1\bar{1}\cdots N\bar{N}\rangle}
\end{eqnarray}
The denominator will be unity if the wave function is
properly orthonormalized. The numerator can be broken into
one-electron and two-electron terms, where the one-electron terms are
given by
\begin{equation}
  \langle 1\bar{1}\cdots N\bar{N}|h|1\bar{1}\cdots N\bar{N} \rangle
	= \sum_{i=1}^N 2h_{ii}
\end{equation}
and the two-electron terms are given by
\begin{equation}
  \langle 1\bar{1}\cdots N\bar{N}|v^{HF}|1\bar{1}\cdots N\bar{N} \rangle
	= \sum_{i,j=1}^N (2J_{ij}-K_{ij})
\end{equation}

The electronic energy is given by
\begin{equation}
  E_{el} = \sum_i^N 2h_{ii} + \sum_{ij}^N(2J_{ij}-K_{ij})
\label{eq:eel}
\end{equation}
The $J_{ij}$ terms are matrix elements of the Coulomb
operator, which is the quantum mechanical operator corresponding to
the macroscopic Coulombic repulsion between electrons $i$ and $j$. The
one-particle Coulomb operator $J^i(1)$ is given by
\begin{equation}
  J^i(1) = \int\frac{\phi_i^*(2)\phi_i(2)}{r_{12}}dr_2
\label{eq:jone}
\end{equation}
where $r_{12}$ is the distance between electrons 1 and
2. The matrix element $J_{ij}$ is given by
\begin{equation}
  J_{ij} = \int\phi_j^*(1)J^i(1)\phi_j(1)dr_1 
	= \int\phi_i^*(1)J^j(1)\phi_i(1)dr_1 
\end{equation}
This element is commonly written $(ii|jj)$, where the first
half of the symbol corresponds to electron 1 and the second part of
the symbol corresponds to electron 2.

The $K_{ij}$ terms are elements of the exchange operator, which is
purely a manifestation of the anti-symmetry of the wave function and
has no macroscopic correspondence. The one-particle exchange operator
$K^i(1)$ is most easily defined in terms of its action on another
orbital $\phi_j$:
\begin{equation}
  K^i(1)\phi_j(1) = \left(\int\frac{\phi_i^*(2)\phi_j(2)}
		{r_{12}}dr_2\right)\phi_i(1).
\label{eq:kone}
\end{equation}
The $K_{ij}$ matrix element is given by
\begin{equation}
  K_{ij} = \int\phi_j^*(1)K^i(1)\phi_j(1)dr_1 
	= \int\phi_i^*(1)K^j(1)\phi_i(1)dr_1 
\end{equation}
This matrix element is often written as $(ij|ij)$.

\section{The Fock Equations}
\label{sec:fock}
The variational principle states that the energy evaluated via
\ref{eq:eel} of any approximate wave function is an upper
bound to the exact energy. Therefore, the optimal
orbitals ${\phi_i}$ are those that give the lowest energy of
the total wave function. As orbital $\phi_i$ changes to $(\phi_i
+ \delta\phi_i) = (\phi_i + \delta)$, the electronic energy from
\ref{eq:eel} changes to
\begin{equation}
  E(i+\delta) = E(i) + 4\sum_i^N\langle\delta|F^i|i\rangle
	+ \mathcal{O}(\delta^2)
\end{equation}
where $F^i$ is the \emph{Fock operator} given by
\begin{equation}
  F^i = h + J^i + \sum_{j\neq i}(2J^i-K^i)
\label{eq:fock}
\end{equation}
The Fock operator corresponds to the first derivative of the
electronic energy with respect to variations in the orbitals.  Because
$J_{ii} = K_{ii}$,
\begin{equation}
  J_{ii} = 2J_{ii} - K_{ii}
\end{equation}
and we can add and subtract self-terms to obtain the
closed-shell Fock operator
\begin{equation}
  F^c = h + \sum_j (2J^j-K^j)
\end{equation} 
\begin{equation}
  F^c_{ij} = \langle i|F^c|j \rangle
\end{equation}
which is the same for all orbitals in our doubly-occupied
core. It is easy to show that variations between occupied orbitals do
not change the electronic energy for the closed-shell wave function
being considered, and consequently the orbital variations
$\delta\phi_i$ must be orthogonal to all occupied orbitals.

\section{Basis Set Expansions}
\label{sec:basis}
In practice, the orbital optimization is achieved by expanding the
orbitals in a set of Gaussian basis functions
${\chi_\mu}$ 
\begin{equation}
  \phi_i = \sum_\mu^{N_{bf}} c_{\mu i}\chi_{\mu}
\label{eq:bset}
\end{equation}
for a basis set with $N_{bf}$ basis functions. Using
Gaussian basis functions both one-electron and two-electron integrals
are easily evaluated.
\begin{equation}
  h_{ij} = \sum_{\mu\nu}^{N_{bf}}c_{\mu i}c_{\nu j}h_{\mu\nu}
\label{eq:hij}
\end{equation}	
is the expression for the $h_{ij}$ matrix element, where
$h_{\mu\nu}$ is the one electron operator element between basis
functions $\chi_\mu$ and $\chi_\nu$, and
\begin{equation}
  J_{ij}^k = \langle i|J^k|j \rangle 
	= (kk|ij) 
	= \sum_{\mu\nu}^{N_{bf}} c_{\mu i}c_{\nu j}(kk|\mu\nu) 
	= \sum_{\mu\nu}^{N_{bf}} c_{\mu i}c_{\nu j}
	\sum_{\sigma\eta}^{N_{bf}}D_{\sigma\eta}^k 
	(\sigma\eta|\mu\nu)
\label{eq:jij}
\end{equation}
is the expression for the $ij$-th element of the $J^k$ Coulomb
operator and
\begin{equation}
  K_{ij}^k = \langle i|K^k|j \rangle 
	= (ki|kj) 
	= \sum_{\mu\nu}^{N_{bf}} c_{\mu i}c_{\nu j}(k\mu|k\nu) 
	= \sum_{\mu\nu}^{N_{bf}} c_{\mu i}c_{\nu j}
	\sum_{\sigma\eta}^{N_{bf}}D_{\sigma\eta}^k
	(\sigma\mu|\eta\nu)
\label{eq:kij}
\end{equation}
is the expression for the $ij$-th element of the $K^k$
exchange operator. The terms
\begin{equation}
  (\sigma\eta|\mu\nu) = \int\frac{\chi_\sigma^*(1)\chi_\eta(1)
	\chi_\mu^*(2)\chi_\nu(2)}{r_{12}}dr_1dr_2
\end{equation}
are the two-electron integrals over basis functions, and 
\begin{equation}
  D_{\sigma\eta}^k = c_{\sigma k}c_{\eta k}
\label{eq:dens}
\end{equation}
is the corresponding density matrix element for orbital
$\phi_k$. The set of orbitals are varied by varying the coefficients
$c_\mu i$ of the basis functions.

A wave function of the form of \ref{eq:prodwf} that contains only
doubly occupied orbitals is called a closed-shell wave function. For
closed-shell wave functions, the orbitals are optimized by first
forming the closed-shell Fock operator $F^c$, given by
\ref{eq:fock}, and is obtained by first forming the core density
matrix $D_c$ 
\begin{equation} 
  D_{\sigma\eta}^c = \sum_i^{occ}c_{\sigma i}c_{\eta i}
\end{equation}
where the summation occurs only over doubly-occupied
orbitals. $F^c$ is given by
\begin{equation}
  F_{\mu\nu}^c = h_{\mu\nu} + \sum_{\sigma\eta}^{N_{bf}}
	D_{\sigma\eta}^c\left(2(\mu\nu|\sigma\eta) -
	(\mu\sigma|\nu\eta)\right)
\end{equation}
in basis-function space, and
\begin{equation}
  F_{ij}^c = \sum_{\mu\nu}^{N_{bf}}c_{\mu i}c_{\nu j}F_{\mu\nu}^c
\end{equation}
over molecular orbitals, where $i$ and $j$ now range over
both occupied and virtual (unoccupied) orbitals. The Fock matrix in
molecular orbital space is diagonalized
\begin{equation}
	U^\dagger F U = \epsilon
\end{equation}
and the eigenvectors ${U_i}$ give the linear combination of
occupied and virtual orbitals that give the improved set of orbitals
${\phi_i}$, and the eigenvalues $\epsilon_i$ give the orbital energies
for these orbitals. This procedure is repeated iteratively until
either the orbitals or the energy stops changing; at this point the
optimal set of orbitals has been obtained and the wave function is
said to be converged.

\section{Properties from HF Calculations}
\label{sec:hfprops}
\subsection{Energies and Structures}
The fundamental quantity we get from quantum chemistry calculations is
the \emph{total energy} of the molecular configuration. Suppose we
wanted to be able to estimate the energy of different C$_2$H$_2$Cl$_2$
configurations: Cl$_2$CCH$_2$, \emph{cis}-CHCl-CHCl, and
\emph{trans}-CHCl-CHCl. We could guess fairly accurate geometries for
each configuration, and then use some quantum chemistry technique
(e.g. B3LYP/6-31G**) to compute single-point energies
(i.e. with no geometry optimization) for each structure. By comparing
the energies, we could get a qualitative understanding of the energy
ordering of the different configurations.

But there simply isn't much reason to do guess geometries. As we saw
earlier, it is relatively straightforward to compute the forces and
use these to optimize our guess geometries. We can then obtain both
structural and energetic information about our set of molecules.

When we use a decent quantum chemistry technique, such as
B3LYP/6-31G**, we can expect to compute \emph{relative} energies of
different compounds to roughly 5 kcal/mol. There are a few systems
that might be slightly more reliable (simple alkanes and alkenes), and
a few systems that might be slighly less reliable (often systems
involving different spin states). Transition states typically involve
slightly greater errors in the energies than do ground state
calculations. Geometries are typically reliable to $>$ 0.05 \AA.

\subsection{Atomic Charges}

We know that molecules consist of localized positively charged nuclei,
with a delocalized cloud of electron density glueing them
together. For many applications, particularly in describing molecules
with a classical \emph{force-field}, it is useful to be able to
simplify this electrostatic distribution into a set of charges that
are centered only at the nuclei of the molecules. 

The simplest way to obtain atomic charges is through \emph{Mulliken
charges}. Mulliken charges assume that the contributions to the atomic
charges are determined by the character of the orbitals on the basis
functions. Thus, the charge on atom $A$ is given by:
\begin{equation}
	q_A = Z_A - \sum_i^{N_{orbs}} \sum_{\mu\in A} \sum_{\nu\in A} 
		c_{\mu i}c_{\nu i}S_{\mu\nu}
\end{equation}
where $Z_A$ is the nuclear charge on atom A, $c_{\mu i}$ is
the orbital coefficient on basis function $\chi_\mu$, $S_{\mu\nu}$ is
the amount that the $\chi_\mu$ and $\chi_\nu$ basis function overlap
each other, and the $\mu\in A$ notation signifies that the summation
occurs over only those basis functions that have their centers on atom
$A$.

Unfortunately, the quality of the Mulliken fit is typically only as
good as the quality of the basis set itself. For example, it is
possible (though not advisable!) to perform a very good calculation on
water by putting all of the basis function on the oxygen atom, and
none on either of the two hydrogen atoms. In such a case, the Mulliken
charges would predict that all of the electrons would be on the oxygen
atom, which would thus have a charge of -2, whereas the hydrogen atoms
would each have charges of +1. The point of this argument is not that
anyone would ever use such a basis set (they wouldn't, hopefully), but
that calculations with similarly \emph{unbalanced} basis sets are one
major limitation of Mulliken population analysis. Furthermore, some
scientists use \emph{plane wave basis sets}, which are beyond the
scope of the current discussion, but where the basis functions do not
have origins on \emph{any} atoms.

Because of these limitations, many have begun to use
\emph{electrostatic potential} (ESP) fitting to obtain atomic
charges. The idea behind this technique is to compute the
electrostatic field due to the nuclei and electron on a series of
points outside of the molecule; then a \emph{least-squares fit} is
performed to find the set of atomic charges that best reproduce this
field on the grid of points. There is a certain theoretical elegance
to such an approach, in that it is independent of the basis
set. However, ESP-based charges can display bizzare irregularities of
their own. Since the least-squares fitting procedure is an
overdetermined problem, there are typically many different sets of
charges that can reproduce a given electrostatic field, and there is
no reason to think that the one that the least-squares fitting
procedure obtains is necessarily the proper one.

Right now our best recommendation is to use Mulliken charges.

\section{Suggestions for Further Reading}
Roothan's 1951 review article \emph{New Developments in
Molecular Orbital Theory} \cite{Roothan51} described the Hartree-Fock
approximation and the LCAO approach used here. Szabo and Ostlund's
\emph{Modern Quantum Chemistry} \cite{Szabo82} also provides a
detailed description of this material.

