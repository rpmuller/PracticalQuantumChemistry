\chapter{Electron Correlation: Beyond HF Theory}

\section{Introduction}
Hartree-Fock theory is appropriate for many different applications in
electronic structure theory, especially for the ground states of
molecules near their equilibrium geometries. It does have two major
shortcomings: it ignores much of the electron correlation, and excited
states are difficult to calculate.  

The first shortcoming of HF theory is that electron correlation is
ignored \cite{Szabo82}, except in an averaged sense. Electron
correlation is the interactions between the motions of the individual
electrons, and HF theory, because it calculates an electron's
motion in the average field produced by the other electrons rather
than the exact positions of the other electrons, leaves much of the
electron correlation out. 

One manifestation of the incorrect treatment of electron correlation
in HF theory is the well-known fact that HF wave
functions of the type of equations \ref{eq:prodwf} or \ref{eq:wfngen} do not
dissociate properly. Consider the H2 molecule
described by two basis functions, $\chi_r$ and $\chi_l$. The
ground state HF wave function will be given by 

\begin{equation}
  \Psi_{HF} = |(\chi_r+\chi_l)(\chi_r+\chi_l)\alpha\beta \rangle 
  = \frac{1}{\sqrt{2}}[\chi_r\chi_r + \chi_r\chi_l + \chi_l\chi_r
	+ \chi_l\chi_l](\alpha\beta-\beta\alpha)
\end{equation}

\noindent which is an accurate description of the bonding when the
molecule is near equilibrium bond length. As the molecule dissociates
the terms $\chi_l\chi_l$ and $\chi_r\chi_r$ in the wave function
become increasingly unstable because they correspond to heterolytic
bond cleavage. The wave function should dissociate only to the terms
corresponding to homolytic cleavage, $\chi_l\chi_r$ and
$\chi_r\chi_l$. The valence bond (VB) wave function starts from the
correct dissociation fragments $\chi_l\chi_r$ and $\chi_r\chi_l$.
Consequently, the VB wave function is given by

\begin{equation}
  \Psi_{VB} = |\chi_l\chi_r\alpha\beta+\chi_r\chi_l\alpha\beta
	\rangle 
	= \frac{2}{\sqrt{2}}[\chi_l\chi_r + \chi_r\chi_l]
	(\alpha\beta-\beta\alpha)
\label{eq:vbwfn}
\end{equation}

\noindent which dissociates to the correct limit because it does not
contain the heterolytic cleavage terms $\chi_l\chi_l$ and
$\chi_r\chi_r$. A generalization of the VB wave function, the
generalized valence bond (GVB) wave function, described at length in
the next section, always yields a lower energy than the HF wave
functions, but the difference in energies is negligible for small bond
distances.

Figure \ref{fig:gvb} shows an example of the incorrect dissociation of
HF wave functions. Shown is the dissociation of H$_2$ using a HF wave
function, with respect to the fragment energy of the individual H
atoms. On the same plot is the GVB wave function that will be
discussed in the next section. As is evident from figure
\ref{fig:gvb}, the HF wave function dissociates to an incorrect limit,
one higher in energy than the individual H fragments. The GVB wave
function, on the other hand, dissociates to the correct energy. 

\begin{figure}
\caption{Dissociation of diatomics using HF and GVB wave functions.} 
\label{fig:gvb}
\end{figure}

Another shortcoming of HF theory is that it is often difficult to
converge excited states of wave functions \cite{Foresman92}. Unless
the excited state has a different overall symmetry than the ground
state, it generally collapses to the ground state upon orbital
optimization. This prevents HF theory from providing chemically useful
information about excitation energies and charge densities of excited
states.  

The next section presents a simple correction to some of the HF
shortcomings, the aforementioned GVB wave function, which generalizes
the two electron wave function from equation \ref{eq:vbwfn} for many
electron wave functions.

\section{Generalized Valence Bond Theory}
\label{sec2.6}
One simple correction to HF theory that overcomes many of HF's
shortcomings is Generalized Valence Bond (GVB)
theory \cite{Bobrowicz77,Hunt70,Bair77,Yaffe76}. GVB theory replaces
the closed-shell HF wave function 

\begin{equation}
	\Psi_{HF} = \left|\prod_i^{occ}\phi_i^2\alpha\beta
	\right\rangle
\label{eq:gvbhfwfn}
\end{equation}

\noindent with the GVB wave function \cite{Bobrowicz77}

\begin{equation}
  \Psi_{GVB} = \left|\left[\prod_i^{occ}(\phi_{i1}\phi_{i2}
	+\phi_{i2}\phi_{i1})\right]
	\Theta(1,2,\dots,N_{occ})\right\rangle
\label{eq:gvbwfn1}
\end{equation}

\noindent where $\Theta(1,2,\dots,Nocc)$ is a general spin wave
function for the $N_{occ}$ electrons and the GVB orbitals $\phi_{i1}$
and $\phi_{i2}$ are not orthogonal. It is generally convenient to
replace the general spin coupling $\Theta$ in equation
\ref{eq:gvbwfn1} with the GVB-Perfect Pairing (GVB-PP) wave function
where the two electrons in each GVB pair are paired only with each
other. This reduces equation \ref{eq:gvbwfn1} to \cite{Bobrowicz77}

\begin{equation}
   \Psi_{GVB} = \left|\prod_i^{occ}(\phi_{i1}\phi_{i2}+\phi_{i2}\phi_{i1})
	\alpha\beta\right\rangle,
\label{eq:gvbpp0}
\end{equation}

\noindent or

\begin{equation}
   \Psi_{GVB} = \left|\prod_i^{occ}\phi_{i1}\phi_{i2}
	(\alpha\beta-\beta\alpha)\right\rangle.
\label{eq:gvbpp}
\end{equation}

\noindent Equation \ref{eq:gvbpp} may be regarded as a generalization
of equation \ref{eq:gvbhfwfn}, where each orbital $\phi_i$ in equation
\ref{eq:gvbhfwfn} is replaced by a GVB pair consisting of two
non-orthogonal orbitals $\phi_{i1}\phi_{i2}$

\begin{equation}
	\phi_i\phi_i\alpha\beta \rightarrow 
	\phi_{i1}\phi_{i2}(\alpha\beta - \beta\alpha) 
	= (\phi_{i1}\phi_{i2} + \phi_{i2}\phi_{i1})\alpha\beta.
\end{equation}

For computational purposes, it is convenient to replace the GVB pair 

\begin{equation}
	\left(\phi_{i1}\phi_{i2} + \phi_{i2}\phi_{i1}\right)
	\alpha\beta
\end{equation}

\noindent with the natural orbital representation \cite{Bobrowicz77}

\begin{equation}
	\left(c_{ig}\phi_{ig}\phi_{ig}
	+c_{iu}\phi_{iu}\phi_{iu}\right)\alpha\beta
\end{equation}

\noindent where $\phi_{ig}$ and $\phi_{iu}$ are now orthogonal and
given by

\begin{equation}
   \phi_{i1} = \frac{c_{ig}^{1/2}\phi_{ig}+c_{iu}^{1/2}\phi_{iu}}
	{\sqrt{c_{ig}+c_{iu}}}
\end{equation}

\begin{equation}
   \phi_{i2} = \frac{c_{ig}^{1/2}\phi_{ig}-c_{iu}^{1/2}\phi_{iu}}
	{\sqrt{c_{ig}+c_{iu}}}.
\end{equation}

\noindent With the strong orthogonality constraint, which assumes that GVB
orbitals of different pairs are orthogonal, the energy may
once again be written in the familiar form

\begin{equation}
   E_{el} = \sum_i^{N_{occ}}2f_ih_{ii} + \sum_{ij}^{N_{occ}}
	(a_{ij}J_{ij} + b_{ij}K_{ij})
\end{equation}

\noindent except now $f_i$ is given by \cite{Bobrowicz77} 

\begin{equation}
	f_i = \left\{\begin{array}{ll}
		1 & \mbox{$\phi_i$ is doubly-occupied} \\
		1/2 & \mbox{$\phi_i$ is singly-occupied} \\
		c_i^2 &   \mbox{$\phi_i$ is a pair orbital with} \\
		      & \mbox{GVB CI coefficient $c_i$}.  
		\end{array}\right. .
\end{equation}

\noindent Similarly, 

\begin{equation}
	a_{ij} = 2f_if_j 
\end{equation}

\begin{equation}
	b_{ij} = -f_if_j 
\end{equation}

\noindent except that $b_{ij} = -1/2$ if $\phi_i$ and $\phi_j$ are
both singly-occupied.

Furthermore, if $\phi_i$ is a pair orbital

\begin{equation}
	a_{ii} = f_i
\end{equation}

\begin{equation}
	b_{ii} = 0
\end{equation}

\noindent and if $\phi_i$ and $\phi_j$ are in the same pair,

\begin{equation}
	a_{ij} = 0
\end{equation}

\begin{equation}
	b_{ij} = -c_ic_j
\end{equation}

Because equations \ref{eq:emod}--\ref{eq:gamma} in chapter
\ref{chap:simplehf} are derived based on the general energy expression in
equation \ref{eq:eos}, the orbital optimization equations are still
appropriate for our modified definitions of $f_i$, $a_{ij}$, and
$b_{ij}$. Thus, the same equations that were used to optimize
open-shell HF wave functions can be used to optimize GVB-PP wave
functions.

For a GVB wave function of the form of equation \ref{eq:gvbpp} with
$N_p$ pairs and $2N_p$ natural orbitals (often referred to as a ``GVB
$N_p/2N_p$'' wave function), $2N_p$ different values of $f_i$ are
obtained, and hence the wave function is said to have $2N_p$ shells.
$2N_p$ Fock operators must also be formed, and hence the wave function
is said to also have $2N_p$ Hamiltonians.

The coefficients $c_{ig}$ and $c_{iu}$ for the GVB orbital are
optimized each iteration \cite{Bobrowicz77, Bair77} by solving a
two-by-two configuration interaction (\emph{vide infra}) for each GVB
pair to minimize the overall energy with respect to the $c_{ig}$ and
$c_{iu}$ coefficients. This amounts to solving \cite{Bobrowicz77}

\begin{equation}
	Y^ic_i = c_iE_i
\label{eq:gvbcoef}
\end{equation}

\noindent where

\begin{equation}
  Y^i =  \left[\begin{array}{cc}
	Y^i_{gg} & Y^i_{gu} \\
	Y^i_{gu} & Y^i_{uu} 
	\end{array} \right]
\end{equation}

\begin{equation}
  Y^i_{gu} = K_{ig,iu}
\end{equation}

\begin{equation}
  Y^i_{gg} = \frac{F^{ig}}{f_{ig}}
\end{equation}

\begin{equation}
  Y^i_{uu} = \frac{F^{iu}}{f_{iu}}
\end{equation}

\noindent The added functional freedom associated with having a pair
of orbitals describe the electron pair allows the GVB-PP wave function
to incorporate the appropriate amount of ionic and covalent character
for any particular internuclear separation. Such a modification to the
wave function for the electron pair is tantamount to including
electron correlation between the two electrons in the GVB pair. One
important result is that GVB-PP wave functions dissociate to the
correct limits, yielding accurate physical data for chemical systems.
Moreover, by selecting a higher root to equation \ref{eq:gvbcoef}
excited states can be selected \cite{Foresman92}.

Generally, molecular wave functions have a combination of orbitals
described by (closed- and open-shell) HF and GVB wave functions. The
next section summarizes equations for these wave functions.

\subsection{Summary of Equations for General HF/GVB Wave Functions}
\label{sec2.7}
The general wave function composed of $N_c$ doubly-occupied core
orbitals, $N_o$ singly-occupied open-shell orbitals, and $N_p$ pairs
of variably-occupied GVB natural orbitals is given by

\begin{equation}
	\Psi = |\Psi_{Core}\Psi_{Open}\Psi_{Pair}\rangle
\label{eq:wfngen2.7}
\end{equation}

\noindent where

\begin{equation}
	\Psi_{Core} = \prod_{i=1}^{N_c} \phi_i\phi_i\alpha\beta
\end{equation}

\begin{equation}
	\Psi_{Open} = \prod_{i=1}^{N_o} \phi_i\alpha
\end{equation}

\begin{equation}
	\Psi_{Pair} = \prod_{i=1}^{N_p} (c_{ig}\phi_{ig}^2
		+ c_{iu}\phi_{iu}^2)(\alpha\beta-\beta\alpha)
\end{equation}

\noindent and $c_{ig}$ and $c_{iu}$ are optimized each iteration via
equation \ref{eq:gvbcoef}.

The electronic energy of this general wave function is given by 

\begin{equation}
  E = \sum_{i=1}^{N_{occ}} 2f_ih_{ii} + 
  \sum_{i,j=1}^{N_{occ}} (a_{ij}J_{ij} + b_{ij}K_{ij}).
\label{eq:egen2.7}
\end{equation}

\noindent Here

\begin{equation}
  N_{occ} = N_c + N_o + 2N_p,
\end{equation} 

\noindent $h$, $J$, and $K$, are the standard one- and two-electron
operators given by equations \ref{eq:hij}, \ref{eq:jij}, and
\ref{eq:kij}.

Once again, $f_i$ is given by \cite{Bobrowicz77} 
\begin{equation}
	f_i = \left\{\begin{array}{ll}
		1	& \mbox{$\phi_i$ is doubly-occupied} \\
		1/2	& \mbox{$\phi_i$ is singly-occupied} \\
		c_i^2 	& \mbox{$\phi_i$ is a pair orbital with} \\
			& \mbox{GVB CI coefficient $c_i$}.  
		\end{array}\right.,
\end{equation}

\begin{equation}
	a_{ij} = 2f_if_j,
\end{equation}

\begin{equation}
	b_{ij} = -f_if_j,
\end{equation}

\noindent except that $b_{ij} = -1/2$ if $\phi_i$ and $\phi_j$ are both
singly-occupied. Furthermore, if $\phi_i$ is a pair orbital

\begin{equation}
	a_{ii} = f_i
\end{equation}

\begin{equation}
	b_{ii} = 0
\end{equation}

\noindent and if $\phi_i$ and $\phi_j$ are in the same pair,

\begin{equation}
	a_{ij} = 0
\end{equation}

\begin{equation}
	b_{ij} = -c_ic_j
\end{equation}

Because the general wave function in equation \ref{eq:wfngen2.7}
requires $(2+2N_p)$ different values of $f_i$, this wave function is
said to have $(2+2N_p)$ shells. Furthermore, because the general wave
function requires $(1+N_o+2N_p)$ different Fock operators, the general
wave function is said to have $(1+N_o+2N_p)$ Hamiltonians. These
numbers assume that there are both core and open-shell orbitals. If
there are no open-shell orbitals but there are core orbitals, the
number of shells and Hamiltonians is $(1+2N_p)$; similar changes are
made when there are no core orbitals.

To optimize the orbitals of this general wave function, first the
optimal mixing of the occupied orbitals with the other
occupied orbitals is calculated. The optimal mixing is
determined by first calculating the $\Delta$ matrix,  

\begin{equation}
  \Delta = \left[\begin{array}{cc}
	0 & \frac{A_{ij}}{B_{ij}} \\
	-\frac{A_{ij}}{B_{ij}} & 0
	\end{array}\right]
\label{eq:del2.7}
\end{equation}

\noindent where

\begin{equation}
  A_{ij} = \langle i|F^j-F^i|j\rangle,
\end{equation}

\begin{equation}
  B_{ij} = \langle i|F^j-F^i|i \rangle 
	- \langle j|F^j-F^i|j \rangle + \gamma_{ij},
\end{equation}

\begin{equation}
  \gamma_{ij} = 2(a_{ii}+a_{jj}-2a_{ij})K_{ij}
		+ (b_{ii}+b_{jj}-2b_{ij})(J_{ij}+K_{ij})
\label{eq:gamma2.7}
\end{equation}

\noindent The new set of orbitals $\{\phi^{New}\}$ are obtained from
the old set of orbitals $\{\phi^{Old}\}$ via the transformation

\begin{equation}
  [\phi^{New}] = \exp(\Delta)[\phi^{Old}]
\end{equation}

The next step in the wave function optimization is the calculation of
the optimal mixing of the occupied orbitals with the virtual
orbitals. This optimization is done by forming the Fock
operator 

\begin{equation}
  F^i_{\mu\nu} = f_ih_{\mu\nu} + 
	\sum_{k=1}^{N_{ham}} (a_{ik}J^k_{\mu\nu}+b_{ik}K^k_{\mu\nu})
\label{eq:f2.7}
\end{equation}

\noindent One Fock operator is required for all of the core orbitals,
and another is required for each open-shell and GVB pair orbital. The
Fock operator is transformed into molecular orbitals via equation
\ref{eq:fij}, and diagonalized. The eigenvectors yield the optimal
linear combination of occupied and virtual orbitals to form the new
set of occupied orbitals. The optimization process is repeated until
the orbitals stop changing.

\section{Configuration Interaction}
\label{sec2.8}
Another method of correcting for the shortcomings of HF theory is
Configuration Interaction \cite{Szabo82} (CI). CI considers the
interaction of excited wave functions with the ground
state wave function. As with GVB wave functions, the
excited wave functions give additional functional
freedom for the total CI wave function to use to
adjust to find an optimal energy. For a wave function
consisting of $N$ spin orbitals  

\begin{equation}
  \Psi = |12\cdots i\cdots j\cdots N\rangle
\end{equation}

\noindent with unoccupied spin orbitals $\psi_r, \psi_s, \dots$ the
wave function

\begin{equation}
  \Psi_i^r = |12\cdots i\cdots Nr\rangle
\end{equation}

\noindent is a singly excited wave function obtained from taking an
electron out of occupied orbital $\psi_i$ and putting it into orbital
$\psi_r$.  Similarly, the wave function

\begin{equation}
  \Psi_{ij}^{rs} = |12\cdots Nrs\rangle
\end{equation}

\noindent is a doubly excited wave function obtained by taking an
electron out of each of the occupied orbitals $\psi_i$ and $\psi_j$
and putting them into unoccupied orbitals $\psi_r$ and $\psi_s$. In a
similar fashion triply, quadruply, and so on, excited wave functions
may be formed.

CI wave functions can add electron correlation by including a linear
combination of excited wave functions with the ground
state wave function \cite{Szabo82}

\begin{equation}
  \Psi_{CI} = \Psi + \sum_i^{occ} \sum_r^{Virt}C_i^r\Psi_i^r +
	\sum_{ij}^{occ} \sum_{rs}^{Virt}C_{ij}^{rs}\Psi_{ij}^{rs}+\cdots
\end{equation}

\noindent The CI coefficients C are obtained by forming the CI matrix
$A_{IJ}$, whose elements are given by

\begin{equation}
  A_{IJ} = \langle \Psi_I | H | \Psi_J \rangle 
\end{equation}

\noindent where $\Psi_I$ and $\Psi_J$ are any of the ground state or
multiply excited wave functions. Diagonalizing $A_{IJ}$ yields the
coefficients $C$ as the eigenvectors, and the correlated energy as the
lowest eigenvalue. The singly-excited wave functions interact weakly
with the ground state, and consequently the doubly-excited
determinants are the most important for the correlated energy
\cite{Szabo82}.

The CI wave function can also be used to calculate excited states
\cite{Foresman92}. When $I$ and $J$ range over the singly-excited
determinants the eigenvectors of $A_{IJ}$ yield the linear combination
of excited wave functions in the various excited states, and the 
eigenvalues yield the energies of the excited states.

Although the CI wave function does correct the shortcomings of HF
theory, one major drawback is that the elements in $A_{IJ}$ require a
full transformation of the two-electron integrals
$(\mu\nu|\sigma\eta)$. This transformation scales as
$\mathcal{O}(N_{bf}^5)$ where $N_{bf}$ is the number of basis
functions. HF and GVB calculations scale only as
$\mathcal{O}(N_{bf}^4)$, which means that a CI calculation is
significantly more expensive than a HF or GVB calculation for large
molecules. For CI calculations diagonalization of the $A_{IJ}$ matrix
is often a much more computationally intensive process than the
$\mathcal{O}(N_{bf}^5)$ integral transformation, depending on what
levels of excitations are included, and so the work required to
transform the integrals is negligible. Nonetheless, because the CI
calculations scale as at least $\mathcal{O}(N_{bf}^5)$ they are
generally too expensive for large molecules, and methods that do not
require a full integral transformation, such as the general HF/GVB
wave functions described in Section \ref{sec2.7} become more
attractive.

\section{Multi-Configurational Self-Consistent Field Wave Functions}
\label{sec2.9}
Another method of correcting the shortcomings of HF theory is the
Multi-Configurational Self-Consistent Field (MCSCF) method
\cite{Szabo82,Yaffe76}. Whereas a CI wave function merely
diagonalizes the Hamiltonian matrix between various excited
configurations, the MCSCF solve self-consistently for the optimal
orbitals among the excited configurations. Again, the additional
functional degrees of freedom afforded by the excited wave functions
give the total MCSCF wave function the ability to more accurately
adjust to the constraints of the molecule, which allows MCSCF wave
functions to describe correlated wave functions and excited states.
Each iteration the coefficients of the various component wave
functions in the MCSCF wave function are recalculated, and the
orbitals are optimized using these coefficients. The GVB wave
functions described in Section \ref{sec2.6} are a special case of
MCSCF wave function \cite{Bobrowicz77}. The GVB wave function does not
require an integral transformation to compute its energy and optimize
its wave functions, but in general, the MCSCF energy and orbital
optimization equations do require a transformation of the two-electron
integrals. Like the CI wave function, the integral transformation
makes MCSCF calculations prohibitively expensive for large molecules.

\section{Moller-Plesset Perturbation Theory}

\section{Coupled Cluster}
