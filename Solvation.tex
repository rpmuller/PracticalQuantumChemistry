\chapter{Solvation}

\section{Introduction}
This chapter contains a brief discussion of solvation in quantum
chemistry calculations. Chemistry is a science that takes place to a
large degree in water and other solvents. Thus far, we have described
molecules as if they were in a universe unto themselves, with no other
molecules to interact with. This is, of course, a gross
oversimplification. 

Unfortunately, the description of solvents can be computationally
intensive. We first present \emph{explicit solvent models}, which
describe the solvent as a collection of individual molecules. This is
a potentially rigorous approximation, but one that is generally too
expensive for most calculations. We then consider replacing the
explicit solvent molecules with a \emph{dielectric continuum} into
which the molecule is embedded. The dielectric continuum allows
charges in the solute molecule to polarize differently than if the
molecule were alone in a vacuum.

\section{Explicit Solvent Models}

The simplest solvation models simply use collections of the actual
solvent molecules for the solvent interactions. In many ways this
technique is the most accurate way to include a solvent. The full
electronic interaction between the solvent and solute molecules in
explicitly included. However, there are several practical problems
with such an approach. The most serious problem is the computational
expense. Many solvent molecules (100--1000) are required to adequately
solvate a substrate molecule, and the expense of describing these
molecules with quantum chemical techniques generally far outweighs the
effort required to describe the substrate itself. A further difficulty
is that it is difficult to keep the solvent molecules from drifting
off into the vacuum. The solute--solvent complex can be embedded in a
periodic cell, but this technique, in addition to being beyond the
scope of the current discussion, introduces additional problems when
the solute is charged. Finally, adding so many solvent molecules
introduces a large number of degrees of freedom that need to be either
minimized or averaged over during the course of the calculation. For
these reasons it is typically easier to replace the effect of the
solvent molecules with a dielectric continuum.

\section{Dielectric Continuum Models}
The primary effect of the solvent on solute molecules is to act as a
dielectric. This dielectric can interact with the electrostatic field
produced by the molecule to stabilize charge polarization. 

Consider the H-Cl molecule dissociating in the gas phase. With proper
treatment, this molecule will dissociate to form neutral H and Cl
atoms. However, in water solvent HCl will dissociate to
H+ and Cl-, because the dipole moments of the water molecules can
stabilize the H+ and Cl- ions separately.

It is possible to embed solute molecules in a slab of dielectric
material, that can respond in largely the same way that the solvent
molecules would have. The dielectric can be tuned to reproduce the
solvent in two ways. The dielectric constant itself is set to that of
the solvent. Secondly, an effective solvent radius dictates how close
the solvent can approach the solute.

With this information, it is possible to use the
\emph{Poisson--Boltzmann} equation to determine the response of the
solvent to the solute charges.

Typically, a solvation calculation involves computing the gas phase
electronic structure of the solute, fitting that electronic stucture
to a set of atomic charges, solving the Poisson--Boltzmann equation to
determine the solvent field, fitting the solvent field to a set of
charges at the molecular surface, solving the electronic structure of
the solute in the present of the solvent field, and repeating until
the process converges.



\section{Acidity and pKa Information}
It is often useful to compute the pKa of an acidic bond in
a molecule. Quantum chemistry is able to compute these properties, but
there are factors that complicate the matter. Many of these
reactions involve negative ions, and in these cases diffuse functions
must be used to properly describe the anions. Obviously solvation must
be used, and describing a solvated proton is often a bad
approximation, as this species may in fact be properly described by
H$_3$O+, H$_5$O$_2$+, and other similar species. Thus, describing the
seemingly simple dissociation reaction HCl $\rightleftharpoons$ H+ +
Cl- would probably be better described as HCl + H$_2$O
$\rightleftharpoons$ H$_3$O+ + Cl-, and all of the species should be
described in solution, with diffuse functions on the basis set.

